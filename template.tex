\documentclass[a4paper, 14pt]{jreport}
\usepackage{template, bm, color, comment}
\usepackage{amsthm}
\usepackage{here}

\pagestyle{plain}

\makeatletter
\def\@cite#1{$\m@th^{\hbox{\@ove@rcfont #1)}}$}
\makeatother	

\def\chapref#1{\ref{#1}章} % 章参照
\def\secref#1{\ref{#1}節} % 章参照
\def\subsecref#1{\ref{#1}小節} % 節参照

\begin{document}
%----------------------------------------------------------------------
% 表紙
%----------------------------------------------------------------------

\begin{toppage}

  % タイトル
  % 改行を入れる場合には \\ を入力
  % 第2引数はレイアウト調整のためのパラメータ.タイトルと氏名までの間隔を
  % 表している.推奨値は以下の通り.
  %   タイトルが1行の時:9
  %   タイトルが2行の時:8
  %   タイトルが3行の時:7

  \title{タイトルが2行の時にはこうなる\\スタイルファイルのサンプル}{}{8}

  % 所属・氏名
  % 名字と名前の間に全角スペースをいれること
  % \author{ 所属組織 }{ 所属部署 }{ 氏名 }

  \author{〇〇株式会社}{□□部△△課}{× ×}

  % 提出日
  \date{2022}{4}{1}

\end{toppage}

%----------------------------------------------------------------------
% 目次
%----------------------------------------------------------------------
\contents
\pagenumbering{arabic}

%----------------------------------------------------------------------
% 本文
%----------------------------------------------------------------------
\chapter{数式}

\LaTeX で数式を書く時には、文中に含まれる行内数式と、独立した行に置かれる別行立て数式がある。

\section{行内数式}

Albert Einsteinは1907年に、「質量とエネルギーの等価性」とその定量関係を表す、$E=mc^2$という等式を発表した。

上の例のように行内数式を記述するには、数式を\$ で囲う。

\section{別行立て数式}

流体の運動を記述する2階非線形偏微分方程式であるナビエ-ストークス方程式は、以下のように記述される。

\begin{multline*}
  \frac{D \bm{v}}{Dt} = - \frac{1}{\rho}\mathrm{grad}p + \frac{\mu}{\rho}\Delta\bm{v}
  + \frac{\lambda + \mu}{\rho}\mathrm{grad}\Theta +\frac{\Theta}{\rho}\mathrm{grad}(\lambda + \mu)\\
  + \frac{1}{\rho}\mathrm{grad}(\bm{v}\cdot\mathrm{grad}\mu)
  + \frac{1}{\mu}\mathrm{rot}(\bm{v}\times\mathrm{grad}\mu)
  - \frac{1}{\rho}\bm{v}\Delta\mu + \bm{g}
\end{multline*}

上の例のように別行立て数式を記述するには、数式を\verb|\|\{begin\}equationと\verb|\|\{end\}equationで囲う。

%----------------------------------------------------------------------
%参考文献の記入欄
%----------------------------------------------------------------------
\bibliographystyle{junsrt}
\bibliography{template}

%----------------------------------------------------------------------
%以下,図表のためのページ
%再びページ番号を振り直しています.
%論文が完成すると,本文と図表の間に色紙を挟みます. 
%----------------------------------------------------------------------

% 付図・付表のリストを作成する
%\appendixpage{1}

\end{document}